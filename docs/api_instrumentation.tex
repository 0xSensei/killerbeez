The instrumentation modules are what track the state of a process and
determine if a path through the process is new.  This will include things such
as QEMU (for Linux), LLVM (for source), PIN, Dynamo-RIO, Dyninst, and Intel PT.
They are optionally given some state information.  The state information is
module-specific and is used to tell the instrumentation module which paths have
been previously hit.  Additionally, each instrumentation module will have a
variety of configuration options that can be specified that will be
specific to that instrumentation module.  These options will be specified as a
JSON char array.

\par
Anything which is instrumentation specific will only be used within the
instrumentation functions.  All other components will treat these items as
opaque strings/blobs.

\api{void * create(char * options, char * state)
}{
This function will create and return an instrumentation struct that defines
the instrumentation's state.  The state argument will be used to load the
previously executed paths through the fuzzed program.
}{
\item options - a JSON string that contains the instrumentation specific options
\item state - used to load a previously dumped state (produced by the get\_state()
function), that defines the current paths seen by the instrumentation.
Alternatively, NULL can be provided to start an instrumentation without a
previously dumped state
\item return value - an instrumentation specific structure or NULL on failure.
The returned value should not be used for anything other than passing to the
various Instrumentation API functions
}

\api{void cleanup(void * instrumentation\_state)
}{
This function will release any resources that the instrumentation has open and
free the instrumentation state.
}{
\item instrumentation\_state - an instrumentation specific structure previously
created by the create function.  This structure will be freed and should not be
referenced afterwards
}

\api{char * get\_state(void * instrumentation\_state, int *out\_length)
}{
This function will return the state information holding the previous execution
path info.  The returned value can later be passed to the instrumentation
create() function to load the state back into an instrumentation struct.
It is the caller's responsibility to free the memory allocated and returned
here using the free\_state() function.
}{
\item instrumentation\_state - an instrumentation specific structure previously
created by the create function
\item out\_length - this pointer will be filled with the length of the returned
state buffer
\item return value - a buffer that holds information about the previous
execution paths as a JSON char array.
}

\api{void free\_state(char * state)
}{
This function will free a previously dumped state (via the get\_state()
function) of the instrumentation.
}{
\item state - a previously dumped state buffer obtained by the get\_state()
function
}

\api{int set\_state(void * instrumentation\_state, char * state)
}{
This function will set the previous execution paths of the instrumentation.
This can be used to restart an instrumentation once it has been created.
}{
\item instrumentation\_state - an instrumentation specific structure previously
created by the create() function
\item state - a previously dumped state buffer obtained by the get\_state()
function
\item return value - 0 on success or non-zero on failure
}

\api{void * merge(void * instrumentation\_state,
void * other\_instrumentation\_state)
}{
This function will merge two sets of instrumentation coverage data.  The
resulting instrumentation state will include the tracked coverage from both
instrumentation states.  Both instrumentation states must have the same
instrumentation options (what to track coverage of, which modules, etc.)
specified, and generally need to be produced by the same instrumentation
module in order for the merge to work correctly.  It's possible that two
different instrumentation modules may produce state information in the same
format, however this is up to them and not something guaranteed by this
specification.  Neither argument will be modified nor freed.  It is the caller's
responsibility to free the memory allocated and returned here using the
free\_state() function.
}{
\item instrumentation\_state - an instrumentation specific structure previously
created by the create() function
\item other\_instrumentation\_state - a second instrumentation specific structure
previously created by the create() function that should be merged with the first
\item return value - an instrumentation specific structure that combines the
coverage information from both of the instrumentation states or NULL on failure
}

\api{int enable(void * instrumentation\_state, HANDLE * process, char * cmd\_line,
char * input, size\_t input\_length)
}{
This function will enable the instrumentation module for a specific process and
runs that process.  If the process needs to be restarted, it will be.
}{
\item instrumentation\_state - an instrumentation specific structure previously
created by the create() function
\item process - a pointer to a handle for the process on which the
instrumentation was enabled
\item cmd\_line - the command line of the fuzzed process on which to enable
instrumentation
\item input - pointer to the buffer containing the input data that should be
sent to the fuzzed process
\item input\_length - the length of the input parameter
\item return value - 0 on success, non-zero on failure
}

\api{int is\_new\_path(void * instrumentation\_state, int * process\_status)
}{
This function will determine whether the process being instrumented has taken a
new path.  It should be called after the process has finished processing the
tested input.
}{
\item instrumentation\_state - an instrumentation specific structure previously
created by the create() function
\item process\_status - pointer that will be filled with a value representing
whether the fuzzed process crashed or hung, or neither
\item return value - 1 if the previously setup process (via the enable()
function) took a new path, 0 if it did not, or -1 on failure
}

\api{int get\_module\_info(void * instrumentation\_state, int index, int * is\_new,
char ** module\_name, char **info, int size)
}{
This function is optional and not required for the fuzzer to work.  It
can be used to obtain coverage information for each executable/library
separately.  This function returns information about each of the separate
modules (shared libraries such as .dll, .so, .dynlib).
}{
\item instrumentation\_state - an instrumentation specific structure previously
created by the create() function
\item index - an index into the module list for the module about which
information should be retrieved.  The return value will indicate if a module
exists for this index.  Indices start at 0 and increase from there
\item is\_new - This parameter returns whether or not the last run of the
instrumentation returned a new path for the module with the specified index.  In
order for the information returned in this parameter to be accurate, the
is\_new\_path method should be called first.  This parameter is optional and can
be set to NULL, if you do not want this information
\item module\_name - This parameter returns the filename of the module at the
specified index.  This parameter is optional and can be set to NULL, if you do
not want this information.  This parameter should not be freed by the caller
\item info - This parameter returns the per-instrumentation path info for the
module with the specified index.  For example, for the DynamoRIO module, the
returned info is an AFL style bitmap of the edges.  This parameter is optional
and can be set to NULL, if you do not want this information.  This parameter
should not be freed by the caller
\item size - This parameter returns the size of the per-instrumentation path
info in the returned info parameter.  This parameter is optional and can be set
to NULL, if you do not want this information
\item return value - non-zero if module with the specified index cannot be
found, or 0 if it is found
}

\api{instrumentation\_edges\_t * get\_edges(void * instrumentation\_state,
int index)
}{
This function is optional and not required for the fuzzer to work.  It is used
by the tracer.  This function returns an array of basic block edges that
occurred in the most recent run of the instrumentation.
}{
\item instrumentation\_state - an instrumentation specific structure previously
created by the create() function.
\item index - If per-module instrumentation information is enabled, this
parameter is an index into the module list for the module about which edges
should be retrieved.  The return value will indicate if a module exists for this
index.  Indices start at 0 and increase from there.  If per-module
instrumentation information is NOT enabled, then this parameter is ignored and
the general edges array will be returned.
\item return value - NULL if an array of basic block edges was not tracked for
the most recent instrumentation run or per-module instrumentation is enabled and
the requested index was not found.  Otherwise, an instrumentation\_edges\_t pointer
that contains an array of basic block edges that were hit in the most recent
instrumentation run.  The returned pointer should be freed by the caller.
}
