\documentclass{article}
\title{Killerbeez API}
\author{GRIMM}
\date{2017.12.22}
% 1 = definition, 2 = description, 3 = args/return value
\def\api#1#2#3{
\bigskip
% Can't make \texttt bold...
% Maybe use \lstinputlisting ?
\texttt{#1}
\par
#2
\par
\begin{itemize}
#3
\end{itemize}
}


\begin{document}
% Nice cover page
\thispagestyle{empty}
\maketitle
\newpage

% Table of Contents
\tableofcontents
\newpage

\section{Overview}
This document will cover the specific API of each module, along with a quick
high-level summary of what it does.

\par
The APIs are all specified in C, as this provides a consistent language and is
explicit about data types which means there's no need for a separate Python
specification.  The C code is frequently wrapped with Python (via ctypes), but
modules are typically written in C code as they run considerably faster when
it is all native code.

\section{Manager}
The manager is what coordinates a fuzz job.  It decides which seeds to use,
which mutators to run and kicks off one or more Main Fuzzers.

\section{Main Fuzzer}
This will run many iterations of a single seed and a single mutator against a
target program.  For efficiency, this will be run on the same computer (which
means same O/S) as the target binary.  This component will be an executable that
the manager executes on each of the target systems.  The arguments for this
function are defined in the usage function of the fuzzer.

\section{Mutator}
\label{mutator}
The mutator modules are what actually mutate the seed buffers.  These would
include things like a bit flipper, byte munger and so forth.  They are given an
input buffer and optionally some state information.  The state information is
module-specific and allows the mutator to pick up where it left off.  For
example, the bit flipper mutator module, which simply flips one bit in the input
buffer, would just need to record what bit to flip as their state.  On the other
hand, more complicated mutators may need to keep track of more information.
Additionally, each mutator will have a variety of mutator specific configuration
options that can be specified.  Both the mutator state and options will be
specified as JSON char arrays.

\par
Anything which is mutator specific will only be used within the mutator
functions.  All other components will treat these items as opaque strings/blobs.


\api{void init(mutator\_t * m)
}{
This function fills in m with all of the function pointers for this mutator.
% TODO Put this note in a sidebar
Note: This function only appears when compiled as a module.  When
ALL\_MUTATORS\_IN\_ONE is defined, this function will not exist, as there would
be a name collision with all the other init() functions from other modules and
there will not be any need for obtaining this struct, as all the functions will
just be called directly.  It's just the code which uses modules which will want
to use this struct.  ALL\_MUTATORS\_IN\_ONE being defined will cause all the
other functions to have the name of the mutator and an underscore prepended.
This means that the create() function will be called bit\_flip\_create() in the
bit flipper mutator.  The name of the mutator is defined by MUTATOR\_NAME.
}{
\item m - a pointer to a mutator\_t structure that will be filled in with the
function pointers that define this mutator.
\item return value - none
}


\api{void * create(char * options, char * state, char * input,
size\_t input\_length)
}{
This function will allocate and initialize the mutator structure.  The allocated
structure will exist until the cleanup() function is called.
}{
\item options - a JSON string that contains the mutator specific string of
options.
\item state - used to load a previously dumped state (produced by the get\_state()
function), that defines the current iteration of the mutator.  This will be a
mutator specific JSON string.  Alternatively, NULL can be provided to start a
mutator without a previously dumped state.
\item input - base input string which will be modified to produce mutated inputs
later when the mutate() function is called
\item input\_length - the size of the input buffer
\item return value - a mutator specific structure or NULL on failure.  The
returned value should not be used for anything other than passing to the various
Mutator API functions.
}


\api{void cleanup(void * mutator\_state)
}{
This function will release any resources that the mutator has open and free the
mutator state structure.
}{
\item mutator\_state - a mutator specific structure previously created by the
create() function.  This structure will be freed and should not be referenced
afterwards.
}


\api{
int mutate(void * mutator\_state, char * buffer, size\_t buffer\_length)
}{
This function will mutate the input given in the create() function and return it
in the buffer argument.  The size of the buffer will be mutator
specific.  For example, some mutators may require this buffer to be larger than
the original input (passed to the create() function) as it's going to extend the
original input in some way.  Other mutators will want it to be the same size.
Guidance on this will be specified by the mutator specific documentation.
}{
\item mutator\_state - a mutator specific structure previously created by the
create() function.
\item buffer - a buffer to which the mutated input will be written
\item buffer\_length - the size of the passed in buffer argument
\item return value - the length of the mutated data on success, 0 when the
mutator is out of mutations, or -1 on error
}

\api{
int mutate\_extended(void * mutator\_state, char * buffer, size\_t buffer\_length, uint64\_t flags);
}{
This function is identical to the \texttt{mutate} function, with the exception
that it accepts a flags parameter that specifies how the mutations should be
done.
}{
\item mutator\_state - a mutator specific structure previously created by the
create() function.
\item buffer - a buffer to which the mutated input will be written
\item buffer\_length - the size of the passed in buffer argument
\item flags - this parameter is a bitfield that specifies how the mutations
should be done. Available flags are:
\begin{itemize}
\item \texttt{MUTATE\_THREAD\_SAFE} - The mutator should ensure that the mutations
are done in a thread safe way.  If the mutator will be accessed via multiple
concurrent threads, this flag should be set.
\item \texttt{MUTATE\_MULTIPLE\_INPUTS} - If the mutator will be handling
individual input parts, this flag should be used.  For some of the fuzzer
applications, it may be necessary to split the input up into separate pieces
that are mutated independently.  In these cases, the mutator can be given
multiple inputs and asked for mutations of the input parts individually.  One
example user of this API is the network driver, where each input is a separate
network packet sent to the target process.  When this flag is set, the index of
the input part to mutate should be included in the lowest 16-bits of the flags
parameter.  For instance, to mutate the fifth input buffer, set flags to \\
\texttt{(MUTATE\_MULTIPLE\_INPUTS | 5)}.
\end{itemize}
\item return value - the length of the mutated data on success, 0 when the
mutator is out of mutations, or -1 on error
}

\api{char * get\_state(void * mutator\_state)
}{
This function will return the state of the mutator.  The returned value can be
used to restart the mutator at a later time, by passing it to the create() or
set\_state() function.  It is the caller's responsibility to free the memory
allocated here using the free\_state() function.
}{
\item mutator\_state - a mutator specific structure previously created by the
create() function.
\item return value - a buffer that defines the current state of the mutator.
This will be a mutator specific JSON string.
}


\api{void free\_state(char * state)}{
This function will free a previously dumped state (via the get\_state()
function) of the mutator.
}{
\item state - a previously dumped state buffer obtained by the get\_state()
function.
}


\api{int set\_state(void * mutator\_state, char * state)
}{
This function will set the current state of the mutator.  This can be used to
restart a mutator once from a previous run.
}{
\item mutator\_state - a mutator specific structure previously created by the
create() function.
\item state - a previously dumped state buffer obtained by the get\_state()
function.  This will be a mutator specific JSON string.
\item return value - 0 on success or non-zero on failure
}


\api{
int get\_current\_iteration(void * mutator\_state)
}{
This function will return the current iteration count of the mutator, i.e. how
many mutations have been generated with it.
}{
\item mutator\_state - a mutator specific structure previously created by the
create() function.
\item return value - the number of previously generated mutations
}


\api{int get\_total\_iteration\_count(void * mutator\_state)
}{
This function will return the total possible number of mutations with this
mutator.  For some mutators, this value won't be possible to predict or the
mutator will be capable of an infinite number of mutations.
}{
\item mutator\_state - a mutator specific structure previously created by the
create() function.
\item return value - the number of possible mutations with this mutator.  If
this number can't be predicted or is infinite, -1 will be returned.
}


\api{void get\_input\_info(void * mutator\_state, int * num\_inputs, size\_t **input\_sizes)
}{
This function will retrieve the number of inputs and the size of each input that
is managed by a mutator.  For most of the simple mutators, they will only be
given a single input.  However, some of the more complicated mutators, such as
the manager mutator, will manage several input buffers and mutate them
independently with the \texttt{mutate\_extended} function.  This function
will return the number of inputs that a mutator is mutating and the sizes of
each of those inputs.
}{
\item mutator\_state - a mutator specific structure previously created by the
create() function.
\item num\_inputs - a pointer to an integer which will be used to return the
number of inputs that a mutator has.
\item input\_sizes - a pointer to a size\_t array that will be used to return
the size of each of the inputs.
}


\api{int set\_input(void * mutator\_state, char * new\_input,
size\_t input\_length)
}{
This function will set the input (saved in the mutator's state) to something new.
This can be used to reinitialize a mutator with new data, without reallocating
the entire state struct.
}{
\item mutator\_state - a mutator specific structure previously created by the
create() function.
\item new\_input - The new input used to produce new mutated inputs later when
the mutate() function is called
\item input\_length - the size in bytes of the input buffer.
\item return value - 0 on success and -1 on failure
}


\api{int help(char ** help\_str)
}{
This function sets a help message for the mutator. This is useful if the mutator
takes a JSON options string in the create() function.
}{
\item help\_str - A double pointer that will be updated to point to the new help
string.
\item return value - 0 on success and -1 on failure
}


\section{Driver}
\label{driver}
The driver will be the component that runs the program being fuzzed.  The driver
should start the program, feed in the input, and determine when the program is
done processing the input.  This component may need to be customized per
target application.

\par
Anything which is driver specific will only be used within the driver functions.
All other components will treat these items as opaque strings/blobs.


\api{void * create(char * options, instrumentation\_t * instrumentation,
void * instrumentation\_state, mutator\_t * mutator, void * mutator\_state)
}{
This function will allocate and initialize the driver structures.  If the driver
is going to be testing a long-running process, this function should start that
process.  Anything that needs to be done before a fuzzing run can start should
be done here.
}{
\item options - a JSON string that contains the driver specific string of
options.
\item instrumentation - a pointer to an instrumentation instance that the driver
will use to instrument the requested program.  The caller should initialize this
instrumentation instance before the create call to the driver, and then free it
after cleaning up the driver.  This parameter is optional and can be set to NULL
if the caller does not wish to use an instrumentation with the driver.
\item instrumentation\_state - a pointer to the instrumentation state for the
passed in instrumentation.  This parameter is optional and can be set to NULL
if the caller does not wish to use an instrumentation with the driver.
\item mutator - a pointer to a mutator instance that the driver can use to
obtain the next input (for use in the \texttt{test\_next\_input} function).
This parameter is optional and can be set to NULL if the caller does not wish to
use a mutator with the driver.  Without this parameter, the
\texttt{test\_next\_input} and \texttt{get\_last\_input} functions will be
unavailable.
\item mutator\_state - a pointer to the mutator state for the passed in mutator.
This parameter is optional and can be set to NULL if the caller does not wish to
use a mutator with the driver.
\item return value - a driver specific structure or NULL on failure.  The
returned value should not be used for anything other than passing to the various
Driver API functions.
}

\api{void cleanup(void * driver\_state)
}{
This function will kill any processes created by the driver and clean up
anything else that was created to help fuzzing.  It will also free the driver
state.
}{
\item driver\_state - a driver specific structure previously created by the
create function.  This structure will be freed and should not be referenced
afterwards.
}

\api{int test\_input(void * driver\_state, char * buffer, size\_t length)
}{
This function will cause the program being fuzzed to be tested against the given
input.  This function should block execution until the program being fuzzed has
finished processing the given input.
}{
\item driver\_state - a driver specific structure previously created by the
create function.
\item buffer -  the input that should be tested
\item length - the length of the buffer argument
\item return value - 0 on success, -1 on failure
}

\api{int test\_next\_input(void * driver\_state);
}{
This function uses the mutator given during the driver creation to retrieve the
next mutated input and test it against the target program.  This function blocks
execution until the program being fuzzed has finished processing the mutated
input.  This function is only available if a mutator was given to the driver
in the \texttt{create} function.
}{
\item driver\_state - a driver specific structure previously created by the
create function.
\item return value - 0 on success, -1 on failure, or -2 if the mutator has run
out of inputs to to mutate.
}

\api{void * get\_last\_input(void * driver\_state, int * length);
}{
This function retrieves the most recent mutated input that was tested with the
\texttt{test\_next\_input} function.  This function is only available if a
mutator was given to the driver in the \texttt{create} function.
}{
\item driver\_state - a driver specific structure previously created by the
create function.
\item length - a pointer to an integer that will be set to the length of the
returned buffer.
\item return value - on success this function will return a buffer containing
the last input that was tested, or NULL on failure.
}



\section{Instrumentation}
\label{instrumentation}
The instrumentation modules are what track the state of a process and
determine if an path through the process is new.  This will include things such
as QEMU (for Linux), LLVM (for source), PIN, Dynamo-RIO, Dyninst, and Intel PT.
They are optionally given some state information.  The state information is
module-specific and is used to tell the instrumentation module which paths have
been previously hit.  Additionally, each instrumentation module will have a
variety of configuration options that can be specified that will be
instrumentation module specific.  These options will be specified as a JSON char
array.

\par
Anything which is instrumentation specific will only be used within the
instrumentation functions.  All other components will treat these items as
opaque strings/blobs.

\api{void * create(char * options, char * state)
}{
This function will create and return an instrumentation struct that defines
instrumentations state.  The state argument will be used to load the
previously executed paths through the fuzzed program.
}{
\item options - a JSON string that contains the instrumentation specific options
\item state - used to load a previously dumped state (produced by the get\_state()
function), that defines the current paths seen by the instrumentation.
Alternatively, NULL can be provided to start an instrumentation without a
previously dumped state
\item return value - an instrumentation specific structure or NULL on failure.
The returned value should not be used for anything other than passing to the
various Instrumentation API functions
}

\api{void cleanup(void * instrumentation\_state)
}{
This function will release any resources that the instrumentation has open and
free the instrumentation state.
}{
\item instrumentation\_state - an instrumentation specific structure previously
created by the create function.  This structure will be freed and should not be
referenced afterwards
}

\api{char * get\_state(void * instrumentation\_state, int *out\_length)
}{
This function will return the state information holding the previous execution
path info.  The returned value can later be passed to the instrumentation
create() function to load the state back into an instrumentation struct.
It is the caller's responsibility to free the memory allocated and returned
here using the free\_state() function.
}{
\item instrumentation\_state - an instrumentation specific structure previously
created by the create function
\item out\_length - this pointer will be filled with the length of the returned
state buffer
\item return value - a buffer that holds information about the previous
execution paths
}

\api{void free\_state(char * state)
}{
This function will free a previously dumped state (via the get\_state function)
of the instrumentation.
}{
\item state - a previously dumped state buffer obtained by the get\_state
function
}

\api{int set\_state(void * instrumentation\_state, char * state)
}{
This function will set the previous execution paths of the instrumentation.
This can be used to restart an instrumentation once its been created.
}{
\item instrumentation\_state - an instrumentation specific structure previously created by the create function
\item state - a previously dumped state buffer obtained by the get\_state function
\item return value - 0 on success or non-zero on failure
}

\api{void * merge(void * instrumentation\_state,
void * other\_instrumentation\_state)
}{
This function will merge two sets of instrumentation coverage data.  The
resulting instrumentation state will include the tracked coverage from both
instrumentation states.  Both instrumentation states must have the same
instrumentation options (what to track coverage of, which modules, etc.)
specified, and generally need to be produced by the same instrumentation
module in order for the merge to work correctly.  It's possible that two
different instrumentation modules may produce state information in the same
format, however this is up to them and not something guaranted by this
specification.
}{
\item instrumentation\_state - an instrumentation specific structure previously
created by the create function
\item other\_instrumentation\_state - a second instrumentation specific structure
previously created by the create function that should be merged with the first
\item return value - an instrumentation specific structure that combines the
coverage information from both of the instrumentation states or NULL on failure
}

\api{int enable(void * instrumentation\_state, HANDLE * process, char * cmd\_line,
char * input, size\_t input\_length)
}{
This function will enable the instrumentation module for a specific process.
}{
\item instrumentation\_state - an instrumentation specific structure previously
created by the create function
\item process - a pointer to a handle for process that the instrumentation was
enabled on
\item cmd\_line - the command line of the fuzzed process to enable
instrumentation on
\item input - pointer to the buffer containing the input data that should be
sent to the fuzzed process
\item input\_length - the length of the input parameter
\item return value - 0 on success, non-zero on failure
}

\api{int is\_new\_path(void * instrumentation\_state)
}{
This function will determine whether the process being instrumented has taken a
new path.  It should be called after the process has finished processing the
tested input.
}{
\item instrumentation\_state - an instrumentation specific structure previously
created by the create function
\item return value - 1 if the previously setup process (via the enable function)
took a new path, 0 if it did not, or -1 on failure
}

\api{int get\_module\_info(void * instrumentation\_state, int index, int * is\_new,
char ** module\_name, char **info, int size)
}{
This function is optional and not required for the fuzzer to work.  It
can be used to obtain coverage information for each executable/library
separately.  This function returns information about each of the separate
modules (shared libraries such as .dll, .so, .dynlib).
}{
\item instrumentation\_state - an instrumentation specific structure previously
created by the create function
\item index - an index into the module list for the module that information
should be retrieved about.  The return value will indicate if a module exists
for this index.  Indices start at 0 and increase from there
\item is\_new - This parameter returns whether or not the last run of the
instrumentation returned a new path for the module with the specified index.  In
order for the information returned in this parameter to be accurate, the
is\_new\_path method should be called first.  This parameter is optional and can
be set to NULL, if you do not want this information
\item module\_name - This parameter returns the filename of the module at the
specified index.  This parameter is optional and can be set to NULL, if you do
not want this information.  This parameter should not be freed by the caller
\item info - This parameter returns the per-instrumentation path info for the
module with the specified index.  For example, for the DynamoRIO module, the
returned info is an AFL style bitmap of the edges.  This parameter is optional
and can be set to NULL, if you do not want this information.  This parameter
should not be freed by the caller
\item size - This parameter returns the size of the per-instrumentation path
info in the returned info parameter.  This parameter is optional and can be set
to NULL, if you do not want this information
\item return value - non-zero if module with the specified index cannot be
found, or 0 if it is found
}

\api{instrumentation\_edges\_t * get\_edges(void * instrumentation\_state,
int index)
}{
This function is optional and not required for the fuzzer to work.  It is used
by the tracer.  This function returns an array of basic block edges that
occurred in the most recent run of the instrumentation.
}{
\item instrumentation\_state - an instrumentation specific structure previously
created by the create function.
\item index - If per module instrumentation information is enabled, this
parameter is an index into the module list for the module that the edges should
be retrieved about.  The return value will indicate if a module exists for this
index.  Indices start at 0 and increase from there.  If per module
instrumentation information is not enabled, then this parameter is ignored and
the general edges array will be returned.
\item return value - NULL if an array of basic block edges was not tracked for
the most recent instrumentation run or per module instrumentation is enabled and
the requested index was not found.  Otherwise, an instrumentation\_edges\_t pointer
that contains an array of basic block edges that were hit in the most recent
instrumentation run.  The returned pointer should be freed by the caller.
}


%\section{Database}
% TODO: document me!

%\section{Tracer}
% TODO: document me!

%\section{Input Generator}
% TODO: create me and then document me!

\end{document}

