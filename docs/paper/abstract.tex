\textbf{\textit{Abstract}}:
The trend of people increasingly relying on software has continued for
several decades and shows no sign of abating. Businesses rely on Windows and
the applications which run on it, servers are typically some type of UNIX
system, and Apple computers are gaining popularity as of late. The desire is
for these to be stable and resilient to attack, which drives the need to find
software errors which compromise these systems. Many improvements have been
made in the field of software testing, with one of the popular ones being fuzz
testing, or fuzzing for short.  Unfortunately the implementation details make
it difficult to compare or combine different methods, while others are only
available for specific operating systems, or limited to cases where source
code is available.  Killerbeez is intended to pull these technologies together
and get them to interoperate. It is scalable, supports multiple operating
systems, is extensible and will have support for testing both kernel as well
as user space applications. The goal is to be able to measure the effectiveness
of various fuzzing techniques in a variety of situations so the optimal
solution can be applied. 
