The short term focus will be to pull in existing technologies from other
projects and get them integrated with Killerbeez and running on all the
supported platforms.  There is ample research, tools and techniques available
now which have yet to be applied in different domains. Once the state of the
art has been incorporated, more automation will be the next priority.

Expanding the coverage of the instrumentation modules so they work on more
operating systems is a high priority. The \IPT{} module currently only works
on Linux, but should be able to work on Windows now that driver support has
improved. The reverse is true for DynamoRIO, which currently works on Windows
but should be able to be ported to run on macOS and Linux without too much
difficulty.  Pulling in more instrumentation technologies such as Intel's Pin,
and Dyninst\cite{dyninst} is also on the list of things to do. Wrapping up the
future enhancements to instrumentation is making more instrumentations aware of
the non-deterministic portions of code which were identified by the Picker.
Currently only the DynamoRIO instrumentation can use this information, but
it should be easy to extend this to all of the \AFL{}-bitmap compatible
instrumentation modules.  This includes the \AFL{} module and the to-be-written
Pin and Dyninst modules.  The \IPT{} module will not be able to use this data
because it does not actually decode the TNT and TIP bitstreams to determine what
code is being executed. Adding real-time parsing would slow down the target
software significantly, however this would still be considered if the benefit
of handling non-deterministic code appears to be worth the additional overhead
which would be incurred. Finally, adding in the Linux-only instrumentation
technologies from Honggfuzz\cite{honggfuzz} is planned, to get the performance
boost on software which runs on Linux, especially that which is closed source
or not written in either C or C++.

Expanding the selection of drivers to include the ability to monitor dialog
box pop-ups on Windows is an area of interest. There are already drivers for
the more common input methods, such as files, stdin, and network data, however
these should also be expanded to cover kernel functions via syscalls, drivers
via \IOCTLs{}, IPC messages, and so forth. This will make Killerbeez a fuzzer
which can handle not only applications on multiple operating systems, but also
the kernel of multiple operating systems as well.  Currently this is only
theoretically possible with Killerbeez, which is not much help to researchers
in the field. Making it supported without any development or modifications
will be a big step in helping industry researchers.

The mutation algorithms from several other projects have already been
integrated with Killerbeez, however it is desirable to have the ability
to write mutators in Python.  This will allow pulling in mutators from
projects like BrundleFuzz\cite{brundlefuzz} as well as quickly putting
together a custom mutator without needing to learn as many of the Killerbeez
APIs, nor even needing a compiler.

In addition to the aforementioned modules on the client side, there are a number
of algorithms to pull in from academic publications.  This includes the seed
energy rating and power schedules from AFLFast\cite{aflfast} and
AFLGo.\cite{aflgo} FairFuzz presents another seed selection algorithm to bias
seed files toward code segments which are not often executed.\cite{fairfuzz}
The algorithm from the PerfFuzz\cite{perffuzz} paper can be pulled into
Killerbeez to find algorithmic complexity vulnerabilities. The research on
using estimators and extrapolators to determine if a fuzzing campaign should
be stopped, or continue running, can be pulled in from Pythia.\cite{pythia}
Instrumentation modules can be updated to implement the collision resistant
algorithm from CollAFL.\cite{collafl} Angora is also on the list of techniques
to integrate into Killerbeez, however it is at the end of the queue of
improvements due to the fact that it would need to be completely re-implemented
on account of the authors never releasing the code.\cite{angora}

The need to obtain a wide variety of seed values is a weakness of Killerbeez
as well as many other fuzzers. The quality of the starting corpus makes a big
difference of the efficiency of fuzzing. This will be addressed by leveraging
existing tools which find new inputs more efficiently, or are able to find
inputs which reach code that is unlikely to be hit via mutation.

There are a number of security analysis and even fuzzing tools which do not
fit into mutators, drivers, nor instrumentation modules. This includes input
generation tools, such as Driller\cite{driller} and Synfuzz\cite{synfuzz}.
Integrating Driller can be accomplished by simply scheduling \BOINC{} jobs
which run Driller instead of the fuzzer component. The input to the process
is still the executable and the code which has been covered thus far, and the
output is still inputs which cause new code to be executed, so this will fit
right in with the current system.  This allows the \BOINC{} code to be
leveraged to handle things like scheduling tasks, dealing with worker nodes
which time out, and the campaign manager can deal with deciding how much time
to spend drilling versus fuzzing. Preparing a Driller environment which can
be easily deployed will require a bit of work, however this should only need
to be done one time, not once for each fuzzing target.

Pulling in inputs generated with Synfuzz should be even easier, as the only
REST API which should be required is the ability to add files to the corpus.
Some automation of setting up Synfuzz may also be possible, but it requires an
oracle to determine which inputs are valid or not, which will likely mean it
will need to be set up manually, short of a development which allows
autonomously detecting such oracles and hooking the appropriate functions.

Finally, expanded support for fuzzers other than the Killerbeez fuzzer is planned. With
the client/server architecture which was chosen, there's no reason Killerbeez
has to be limited to a single fuzzer.  It could easily just run Honggfuzz,
\AFL{}, WinAFL, or others.  Doing so will require additional configuration in
\BOINC{} to ensure the previous state is sent to the \BOINC{} clients, as well
as a new assimilator which is the \BOINC{} component which gathers the results.

% Not sure if we want to take on the things below or not, but if so they will
% be in the far future.
% Unsolved problems:

% How to choose targets
% Where to get seeds?
%   Search the web, very ad-hoc and manual process
% Avoiding the easy-to-find crashes to get to the more interesting ones
%   Could be done with smarter input generators/mutators & automated static/dynamic analysis
% Targets which include checksums, compression or encryption
%   Typical solution, modify the source/executable to remove those checks


% Room for improvement:

% Detecting non-crashing errors (especially without source code for the target)


