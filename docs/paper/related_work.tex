There are projects which have addressed some of the aspects covered by Killerbeez
such as platform independence, distributed fuzzing, leveraging existing tools,
and so forth.

Google's OSS-Fuzz\cite{ossfuzz} addresses scalability by running many fuzzers
in parallel, as well as re-using existing code by leveraging tools like
Honggfuzz to handle the actual fuzzing. The core fuzzing component of OSS-Fuzz,
CluserFuzz, is unpublished and therefore unavailable to anyone outside of
Google.  Another difference is that OSS-Fuzz requires source code for the
target software, and also requires that tests be written to integrate it
into the overall system. This adds efficiency, as the test cases can eliminate
code like GUI libraries, which is not the real target of the fuzzing, however
it is unable to test closed source software.

\AFL{}\cite{afl} is an open source fuzzer that uses coverage data and genetic
algorithms to automatically discover interesting test cases in a target.  \AFL{}
was designed to be practical; it has a low overhead, is easy to use, and works
against real-world software. As such, \AFL{} has become one of the most popular
fuzzers and many research projects investigate how to improve \AFL{}. Killerbeez
borrows many features from \AFL{}, such as the compiler and QEMU
instrumentations. While Killerbeez has also borrowed \AFL{}'s mutation strategy,
it does not currently include \AFL{}'s mutations which mutate based on coverage
data. Coverage data is used to avoid wasting time mutating portions of the input
that the target does not process. These mutations will be incorporated into
Killerbeez in the future.  While \AFL{} is a great local fuzzer, it is not
easily distributed and cannot easily manage the fuzzing of more complex targets.
\AFL{} does not support applications on Windows, and the \AFL{} fork which does
support Windows, WinAFL, lacks many of the features of \AFL{}.

Peach Fuzzer\cite{peach} now supports distributed fuzzing, modular mutators, and
modules for launching apps, is able to do both file and network-based fuzzing,
and does work on closed source applications.  However, the distributed aspect
is only available in the proprietary version of the fuzzer; it does not exist
in the community edition, which is open source. There is also no feedback loop
for code coverage in either edition.  Instead, the input format needs to be
manually described in XML files, as does the model for the program state. To
alleviate this problem, the company behind Peach Fuzzer is willing to sell
access to the definitions they have created.

Honggfuzz\cite{honggfuzz} is an open source fuzzer which runs on Windows, Linux,
macOS, Android, FreeBSD and NetBSD, all using a single code base. It can handle
closed source applications, long-running applications such as servers, and will
automatically use multiple CPU cores to do fuzzing in parallel. Modularity is
achieved by allowing an external program to do the mutation of inputs. While
Honggfuzz scales nicely on a single machine, it does not have any built in
mechanism to utilize multiple machines.  Using Honggfuzz in a larger framework
such as OSS-Fuzz takes care of this limitation.  In fact, Honggfuzz is a fuzzer
which will be likely integrated into Killerbeez in the future, as described in
section \ref{Future Work}. Honggfuzz has more types of instrumentation than any
other fuzzer, including Killerbeez at the time of writing, however none of
these work on Windows. The \BTS{} and \IPT{} instrumentations are only for
Linux, as is the hardware-based counters instrumentation which tracks the
number of instructions and branches which were executed. There is also compile time
instrumentation, but this is only helpful in the case where source code is
available, and it can be compiled by GCC or LLVM.  It is possible to compile
some C/C++ code for windows using LLVM, but anything which requires Microsoft
Visual Studio to be compiled will not have any instrumentation.  Supporting
Windows does not seem to be a priority for Honggfuzz, as it cannot be compiled
natively, but only via Cygwin.

Honggfuzz is a great tool, which is why a modified version of it was created
which can use all of the Killerbeez modules.\cite{honggfuzzgrimm} Section
\ref{Future Work} describes the Linux instrumentation technologies from
Honggfuzz planned for integration with Killerbeez.
If it is feasible to add the ability to use
Killerbeez instrumentation modules, that is another contribution which will be
made to the Honggfuzz project.
