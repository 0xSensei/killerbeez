There are projects which have addressed some of aspects covered by Killerbeez
such as platform independence, distributed fuzzing, leveraging existing tools,
and so forth.

Google's OSS-Fuzz\cite{ossfuzz} addresses scalability by running many fuzzers
in parallel, as well as re-using existing tools by leveraging things like
Honggfuzz to handle the actual fuzzing. The core fuzzing component of OSS-Fuzz,
CluserFuzz, is closed source and unavailable to anyone outside of Google.
Another difference is that OSS-Fuzz requires source code for the target system
to work, and also requires that tests be written to integrate it into the
overall system. This adds efficiency, as the test cases can eliminate code
like GUI libraries, which is not the real target of the fuzzing, however it
also means not being able to test closed source code.

Peach Fuzzer\cite{peach} now supports distributed fuzzing, modular mutators,
modules for launching apps, is able to do both file and network based fuzzing,
and does work on closed source applications.  However, the distributed aspect
is only available in the proprietary version of the fuzzer; it does not exist
in the community edition, which is open source. There is also no feedback loop
for code coverage.  Instead the input format needs to be manually described in
XML files, as does the model for the program state. To alleviate this problem,
the company behind Peach Fuzzer is willing to sell access to the definitions
they have created.

Honggfuzz\cite{honggfuzz} is an open source fuzzer which runs on Windows, Linux,
macOS, Android, FreeBSD and NetBSD, all using a single code base. It can handle
closed source applications, long running applications such as servers, and will
automatically use multiple CPU cores to do fuzzing in parallel. Modularity is
achieved by allowing an external program to do the mutation of inputs. While
Honggfuzz scales nicely on a single machine, it does not have any built in
mechanism to utilize multiple machines.  Using Hongfuzz in a larger framework
such as OSS-Fuzz takes care of this limitation.  In fact, Honggfuzz is a fuzzer
which will be integrated into Killerbeez in the future, as described in section
\ref{Future Work}. Honggfuzz has more types of instrumentation than any other
fuzzer, including Killerbeez at the time of writing, however none of these work
on Windows. The \BTS{} and \IPT{} instrumentations are only for Linux, as is
the hardware-based counters instrumentation which tracks the number of
instructions and branches which were executed. There is compile time
instrumentation, but this only is helpful in the case where source code is
available, and it can be compiled by GCC or LLVM.  It is possible to compile
some C/C++ code for windows using LLVM, but anything which requires Microsoft
Visual Studio to be compiled will not have any instrumentation.

Honggfuzz is a great tool, which is why a modified version of it was created
which can use all of the Killerbeez modules.\cite{honggfuzzgrimm} Section
\ref{Future Work} describes the instrumentation technologies planned on being
integrated from Honggfuzz. If it is feasible to add the ability to use
Killerbeez instrumentation modules, that is another contribution which will be
made to the Honggfuzz project.
